\documentclass{article} 
\usepackage{polyglossia} 
\usepackage{amsmath}
\usepackage{fontspec} 
\usepackage{lipsum} 
\usepackage[margin=1in]{geometry}
\usepackage{graphicx} 
\usepackage{caption} 
\usepackage{subcaption}
\usepackage{hyperref} 
\usepackage{listing}
\hypersetup{% 
    colorlinks=true, linkcolor=blue, filecolor=magenta,      
    urlcolor=cyan, 
    pdfinfo = {%
        Title = Αναφορά 2ης Εργασίας ΨΕΕ
        Author = {Χρήστος Μάριος Περδίκης},
        Producer = XeLaTeX,
    } 
}

\setmainfont{C059}

\title{Ψηφιακή Επεξεργασία Εικόνας \\ Ανίχνευση Ακμών και Κύκλων}
\date{Εαρινό Εξάμηνο 2024-2025}
\author{Χρήστος-Μάριος Περδίκης 10075 cperdikis@ece.auth.gr}

\begin{document}
\maketitle

Αυτή είναι η αναφορά για την 2η εργασία του μαθήματος Ψηφιακή Επεξεργασία 
Εικόνας. Ο στόχος της εργασίας είναι η ανίχνευση ακμών με τους τελεστές Sobel
και Laplacian of Gaussian και η ανίχνευση κύκλων με βάση τον αλγόριθμο Hough.
O αλγόριθμος Hough είναι τροποποιημένος για να ανιχνεύει κύκλους αντί για 
ευθείες. Ακολουθεί η επεξήση των παραδοτέων συναρτήσεων.

\section{Αρχείο επίδειξης demo.py}

\section{fir\_conv}

\section{sobel\_edge}

\section{log\_edge}

\section{circ\_hough}

\end{document}
