\documentclass{article} 
\usepackage{polyglossia} 
\usepackage{amsmath}
\usepackage{fontspec} 
\usepackage{lipsum} 
\usepackage[margin=1in]{geometry}
\usepackage{graphicx} 
\usepackage{caption} 
\usepackage{subcaption}
\usepackage{hyperref} 
\usepackage{listing}
\hypersetup{% 
    colorlinks=true, linkcolor=blue, filecolor=magenta,      
    urlcolor=cyan, 
    pdfinfo = {%
        Title = Αναφορά 2ης Εργασίας ΨΕΕ
        Author = {Χρήστος Μάριος Περδίκης},
        Producer = XeLaTeX,
    } 
}

\setmainfont{C059}

\title{Ψηφιακή Επεξεργασία Εικόνας \\ Ανίχνευση Ακμών και Κύκλων}
\date{Εαρινό Εξάμηνο 2024-2025}
\author{Χρήστος-Μάριος Περδίκης 10075 cperdikis@ece.auth.gr}

\begin{document}
\maketitle

Αυτή είναι η αναφορά για την 2η εργασία του μαθήματος Ψηφιακή Επεξεργασία 
Εικόνας. Ο στόχος της εργασίας είναι η ανίχνευση ακμών με τους τελεστές Sobel
και Laplacian of Gaussian και η ανίχνευση κύκλων με βάση τον αλγόριθμο Hough.
O αλγόριθμος Hough είναι τροποποιημένος για να ανιχνεύει κύ\-κλους αντί για 
ευθείες. Ακολουθεί η επεξήγηση του demo.py και των παραδοτέων συναρτήσεων.

\section{Αρχείο επίδειξης demo.py}
Στο αρχείο demo.py καλούνται όλες οι παραδοτέες συναρτήσεις για να γίνει η 
μελέτη πάνω στην εικόνα της εκφώνησης. Πρώτα φορτώνεται η εικόνα και μετά 
σμικρύνεται στο μισό της αρχικής της ανάλυσης (χρησιμοποιήθηκε η βιβλιοθήκη
opencv). Η αρχική και η σμικρυσμένη εικόνα εμφανίζονται στην οθόνη, το αναμενόμενο 
αποτέλεσμα είναι να φαίνονται ολόιδιες. Για την προβολή των εικόνων 
χρησιμοποιήθηκε το module matplotlib. Έπειτα καλείται sobel\_edge για ανίχνευση ακμών 
με τελεστή Sobel πέντε φορές για τα κατώφλια $thresholds = \left[50,100,150,
200,250\right]$. Η επεξήγηση της συνάρτησης sobel\_edge βρίσκεται στην 
ενότητα~\ref{sobel}. Καλείται η συνάρτηση log\_edge μία φορά για την ανίχνευση ακμών με
τελεστή Laplacian of Gaussian. Η επεξήγηση της συνάρτησης sobel\_edge βρίσκεται στην 
ενότητα~\ref{log}. Οι δυαδικές εικόνες-αποτελέσματα των δύο συναρτήσεων 
προβάλλονται στην οθόνη μαζί. Επίσης δημιουργείται
ένα plot του αριθμού των σημείων ακμών που βρέθηκαν για διάφορες τιμές κατωφλίων. 

Από τα αποτελέσματα των αλγορίθων μπορούμε να εξάγουμε διάφορα συμπεράσματα. 
Μεταξύ των δυαδικών εικόνων Sobel βλέπουμε ότι όσο μικρότερο είναι το threshold
τόσο πιο καθαρές και έντονες φαίνονται οι ακμές της εικόνας. Για πολύ μικρές τιμές 
π.χ. για $threshold = 50$ εμφανίζεται θόρυβος ο οποίος είναι ανεπιθύμητος, δεν 
αντιστοιχεί σε ακμές. Συγκρίνοντας τις εικόνες Sobel με την εικόνα LoG βλέπουμε
ότι η δεύτερη έχει λεπτότερες ακμές και περισσότερο θόρυβο από τις εικόνες 
Sobel. Αυτό είναι αναμενόμενο, καθώς από τη φύση του η μέθοδος του τελεστή LoG
είναι πιο ευαίσθητη στον θόρυβο, με το αντάλλαγμα ότι μπορεί να ανιχνεύσει πιο 
λεπτές ακμές. Στην παρούσα εφαρμογή φαίνεται ο τελεστής Sobel να κάνει καλύτερη
δουλειά όμως.

Ακολουθεί η ανίχνευση κύκλων με Hough. Για προετοιμασία φτιάχνεται μια δυαδική
εικόνα ακμών με τη χρήση του τελεστή Sobel και κατώφλι $100$. Πειραματικά 
φάνηκε ότι για αυτή την εικόνα η συγκεκριμένη τιμή του κατωφλιού είχε ένα 
αρκετά καλό αποτέλεσμα. Η δυαδική εικόνα αυτή είναι η είσοδος της συνάρτησης 
circ\_hough. Η υλοποίηση του αλγορίθμου Hough για κύκλους επεξηγείται αναλυτικά 
στην ενότητα~\ref{hough}. 
Αρχικοποιούνται οι διαστάσεις του πίνακα Hough (32 για τετμημένη και τεταγμένη
του κέντρου του πιθανού κύκλου και 32 για την ακτίνα του κύκλου), η μέγιστη 
ακτίνα του κύκλου στο μισό της μικρότερης διάστασης της εικόνας και οι ελάχιστες
ψήφοι για να ανιχνευτεί ο κύκλος σε 1000. Για την ώρα το πρόγραμμα επίδειξης 
τρέχει τον αλγόριθμο Hough για κύκλους μόνο για έναν κύκλο. Αφού τρέξει ο
αλγόριθμος και εφόσον ανιχνεύσει κύκλος, αυτός προβάλλεται στην οθόνη πάνω 
από την αρχική εικόνα (τη σμικρυσμένη ασπρόμαυρη εικόνα, όχι τη δυαδική εικόνα
αποτέλεσμα του τελεστή Sobel). Μπορούμε να δούμε ότι η κύκλος ανιχνεύεται σωστά,
όμως με κάποιο μικρό offset στο κέντρο του κύκλου και στην ακτίνα. Αυτό οφείλεται
στην ανακρίβεια λόγω μικρού αριθμού bins που χωρίστηκαν οι συντεταγμένες του 
κέντρου του κύκλου και η ακτίνα.

\emph{ΠΟΛΛΟΙ ΚΥΚΛΟΙ ΓΙΑ ΔΙΑΦΟΡΑ V\_min;} 
\section{fir\_conv}
Η συνάρτηση fir\_conv υλοποιεί δισδιάστατη συνέλιξη μεταξύ μιας grayscale
εικόνας και ενός FIR φίλτρου / πεπερασμένης συνελικτικής μάσκας. Βρίσκεται στο 
αρχείο fir\_conv.py. Προαιρετικά έχει σαν είσοδο και τις αρχές των συντεταγμένων 
της εικόνας και της συνελικτικής μάσκας. Αν δεν δοθεί κάποια είσοδος, η αρχή 
των συντεταγμένων για τη συνελικτική μάσκα είναι το κέντρο της και για την εικόνα
είναι στο ίδιο σημείο με αυτό που ορίστηκε όταν φορτώθηκε η εικόνα σε numpy array
(εφόσον έγινε χρήση opencv, η αρχή των συντεταγμένων της εικόνας είναι η πάνω
αριστερά γωνία). Υπολογίζονται οι διαστάσεις της εικόνας εξόδου της συνέλιξης.
Αν $N_1$ και $N_2$ οι διαστάσεις της εικόνας και $n_1$ και $n_2$ οι διαστάσεις
της μάσκας, η εικόνα εξόδου θα έχει διαστάσεις $(N_1 - n_1 + 1, N_2 - n_2 + 1)$.
Η αρχή των συντεταγμένων της εικόνας εξόδου είναι η αρχή των συντεταγμένων της 
εικόνας + η αρχή των συντεταγμένων της συνελικτικής μάσκας. Έπειτα η συνέλιξη 
πραγματοποιείται ολισθαίνοντας την συνελικτική μάσκα πάνω από την εικόνα εισόδου
και αθροίζοντας τα γινόμενα των στοιχείων της μάσκας με τα αντίτοιχα στοιχεία 
της εικόνας. Η έξοδος είναι η εικόνα εξόδου της συνέλιξης και η αρχή των 
συντεταγμένων αυτής της εικόνας.

\section{sobel\_edge}\label{sobel}
Η συνάρτηση sobel\_edge υλοποιεί την μέθοδο εύρεσης ακμών με τον τελεστή 
Sobel. Βρίσκεται στο αρχείο sobel\_edge.py. Βρίσκει τις ακμές μιας εικόνας
εισόδου και παράγει ως έξοδο μια δυαδική εικόνα που έχει άσπρα pixels στις 
ακμές και μαύρα pixels οπουδήποτε αλλού. Αρχικά ορίζονται οι δύο πίνακες Sobel
για την τεταγμένη και την τετμημένη:

\begin{gather}
    sobel_x = \left[\begin{matrix}
        -1, 0, 1 \\
        -2, 0, 2 \\
         1, 0, 1
    \end{matrix}\right] \\
    sobel_y = \left[\begin{matrix}
        -1, -2, -1 \\
         0,  0,  0 \\
         1,  2,  1
    \end{matrix}\right] 
\end{gather}
Έπειτα χρησιμοποιείται η συνάρτηση fir\_conv για να συνελιχθεί οι $sobel_x$ και
$sobel_y$ με την εικόνα. Με αυτόν τον τρόπο υπολογίζεται το gradient της αρχικής
εικόνας κατά άξονα $x$ και $y$ αντίστοιχα ($grad_x$, $grad_y$). Έπειτα υπολογίζεται 
η νόρμα του συνολικού gradient:

\begin{equation}
    |gradient| = \sqrt{grad_x^2 + grad_y^2}
\end{equation}
Δημιουργείται μια νέα εικόνα ίδιων διαστάσεων με την εικόνα εισόδου στην οποία 
τα pixels που έχουν $|gradient| \geq threshold$ βάφονται άσπρα και όλα τα 
υπόλοιπα βάφονται μαύρα. Αυτή είναι η δυαδική εικόνα εξόδου.

\section{log\_edge}\label{log}
H συνάρτηση log\_edge υλοποιεί την εύρεση ακμών 

%3. log_edge.py
%
%This file implements edge detection using the Laplacian of Gaussian (LoG) approach:
%
    %Purpose: Detects edges by looking for zero crossings in the second derivative
    %Function: log_edge(in_img_array)
    %Steps:
        %Defines a 5×5 Laplacian of Gaussian kernel (normalized)
        %Applies this kernel to the input image using the fir_conv function
        %Calculates gradients in x and y directions using central differences
        %Computes the gradient magnitude
        %Creates a binary output image by checking for zero crossings:
            %For each pixel, examines all 8 neighboring pixels
            %If the center pixel has a different sign from any neighbor (indicating a zero crossing) AND the gradient magnitude exceeds a threshold (10), marks it as an edge (1)
            %Otherwise, marks it as non-edge (0)
        %Returns the binary edge image
%
%The LoG method is good at finding edges and is less sensitive to noise than simple gradient methods.
%4. circ_hough.py
%
%This file implements circle detection using the Hough transform:
%
    %Purpose: Detects circles in binary images
    %Function: circ_hough(in_img_array, R_max, dim, V_min)
    %Steps:
        %Takes a binary input image and parameters for circle detection:
            %R_max: Maximum radius to consider
            %dim: Dimensions for binning/discretization
            %V_min: Minimum votes required to consider a circle
        %Creates a 3D accumulator array (voting space) with dimensions for x, y, and radius
        %For each edge pixel in the input image:
            %For each possible radius:
                %For each possible center position:
                    %Checks if the distance between the edge pixel and this potential center is close to the current radius
                    %If so, increments the vote count for this center-radius combination
        %Finds the bin with maximum votes
        %If the maximum votes exceed the threshold:
            %Extracts the center coordinates and radius from the bin indices
            %Returns these as the detected circle
        %If no circle meets the threshold, returns empty arrays

\section{circ\_hough}\label{hough}
Η υλοποίηση της συνάρτησης circ\_hough βρίσκεται στο αρχείο circ\_hough.py.
Αρχικά γίνεται η διαμέριση των 

\end{document}
