\documentclass{article} 
\usepackage{polyglossia} 
\usepackage{amsmath}
\usepackage{fontspec} 
\usepackage{lipsum} 
\usepackage[margin=1in]{geometry}
\usepackage{graphicx} 
\usepackage{caption} 
\usepackage{subcaption}
\usepackage{hyperref} 
\usepackage{listing}
\hypersetup{% 
    colorlinks=true, linkcolor=blue, filecolor=magenta,      
    urlcolor=cyan, 
    pdfinfo = {%
        Title = Αναφορά 2ης Εργασίας ΨΕΕ
        Author = {Χρήστος Μάριος Περδίκης},
        Producer = XeLaTeX,
    } 
}

\setmainfont{C059}

\title{Ψηφιακή Επεξεργασία Εικόνας \\ Ανίχνευση Ακμών και Κύκλων}
\date{Εαρινό Εξάμηνο 2024-2025}
\author{Χρήστος-Μάριος Περδίκης 10075 cperdikis@ece.auth.gr}

\begin{document}
\maketitle

Αυτή είναι η αναφορά για την 2η εργασία του μαθήματος Ψηφιακή Επεξεργασία 
Εικόνας. Ο στόχος της εργασίας είναι η ανίχνευση ακμών με τους τελεστές Sobel
και Laplacian of Gaussian και η ανίχνευση κύκλων με βάση τον αλγόριθμο Hough.
O αλγόριθμος Hough είναι τροποποιημένος για να ανιχνεύει κύ\-κλους αντί για 
ευθείες. Ακολουθεί η επεξήγηση των παραδοτέων συναρτήσεων.

\section{Αρχείο επίδειξης demo.py}
Στο αρχείο demo.py καλούνται όλες οι παραδοτέες συναρτήσεις για να γίνει η 
μελέτη πάνω στην εικόνα της εκφώνησης. Πρώτα φορτώνεται η εικόνα και μετά 
σμικρύνεται στο μισό της αρχικής της ανάλυσης (χρησιμοποιήθηκε η βιβλιοθήκη
opencv). Η αρχική και η σμικρυσμένη εικόνα εμφανίζονται στην οθόνη, το αναμενόμενο 
αποτέλεσμα είναι να φαίνονται ολόιδιες. Για την προβολή των εικόνων 
χρησιμοποιήθηκε το module matplotlib. Έπειτα καλείται sobel\_edge για ανίχνευση ακμών 
με τελεστή Sobel πέντε φορές για τα κατώφλια $thresholds = \left[50,100,150,
200,250\right]$. Η επεξήγηση της συνάρτησης sobel\_edge βρίσκεται στην 
ενότητα~\ref{sobel}. Καλείται η συνάρτηση log\_edge μία φορά για την ανίχνευση ακμών με
τελεστή Laplacian of Gaussian. Η επεξήγηση της συνάρτησης sobel\_edge βρίσκεται στην 
ενότητα~\ref{log}. Οι δυαδικές εικόνες-αποτελέσματα των δύο συναρτήσεων 
προβάλλονται στην οθόνη μαζί. Επίσης δημιουργείται
ένα plot του αριθμού των σημείων ακμών που βρέθηκαν για διάφορες τιμές κατωφλίων. 
\emph{ΣΥΓΚΡΙΣΗ ΕΙΚΟΝΩΝ SOBEL ΜΕ LOG}.

Ακολουθεί η ανίχνευση κύκλων με Hough. Για προετοιμασία φιτάχνεται μια δυαδική
εικόνα ακμών με τη χρήση του τελεστή Sobel και κατώφλι $100$. Πειραματικά 
φάνηκε ότι για αυτή την εικόνα η συγκεκριμένη τιμή του κατωφλιού είχε ένα 
αρκετά καλό αποτέλεσμα. Η δυαδική εικόνα αυτή είναι η είσοδος της συνάρτησης 
circ\_hough η οποία επεξηγείται αναλυτικά στην ενότητα~\ref{hough}. 
\emph{binning. V\_min. xrhsh circle apo matplotlib. anakriveies logw binning.} 
\section{fir\_conv}

\section{sobel\_edge}\label{sobel}

\section{log\_edge}\label{log}

\section{circ\_hough}\label{hough}

\end{document}
