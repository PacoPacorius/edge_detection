\documentclass{article} 
\usepackage{polyglossia} 
\usepackage{amsmath}
\usepackage{fontspec} 
\usepackage{lipsum} 
\usepackage[margin=1in]{geometry}
\usepackage{graphicx} 
\usepackage{caption} 
\usepackage{subcaption}
\usepackage{hyperref} 
\usepackage{listing}
\hypersetup{% 
    colorlinks=true, linkcolor=blue, filecolor=magenta,      
    urlcolor=cyan, 
    pdfinfo = {%
        Title = Ημερολόγιο 2ης Εργασίας ΨΕΕ
        Author = {Χρήστος Μάριος Περδίκης},
        Producer = XeLaTeX,
    } 
}

\setmainfont{C059}


\title{Ημερολόγιο 2ης Εργασίας ΨΕΕ}
\date{Εαρινό Εξάμηνο 2024-2025}
\author{Χρήστος-Μάριος Περδίκης 10075 cperdikis@ece.auth.gr}

\begin{document}
\maketitle

\section{12/5/25 --- Η Αρχή Των Πάντων}
Διάβασα την εκφώνηση.

\section{13/5/25 --- Η Συνέχεια της Αρχής}
Ο προσωπικός βοηθός μου ο Κλαύδιος μου έδωσε το πρώτο draft κώδικα.
Μπορώ να πω οτι τρέχει απίστευτα αργά\ldots{} Πρόσθεσα μερικά μηνύματα για
να ξέρω πού βρίσκεται το πρόγραμμα κάθε φορά. Αύριο ας ακούσω τι έχει να πει ο
Κλαύδιος για τον κώδικα που μου έδωσε και ας τον δω να τρέχει μια φορά 
ολόκληρος. Έπειτα ας προσπαθήσω να υποδειγματοληπτήσω την εικόνα, ή έστω
να την κάνω χαμηλότερης ανάλυσης (αν έχουν διαφορά αυτά τα δύο, που
νομίζω δεν έχουν) και ας προσπαθήσω να αλλάξω το φόρτωμα της εικόνας
από pillow σε opencv.

\section{14/5/25 --- Σοβαρεύουν τα πράγματα}
Θα προσπαθήσω να κάνω την αρχική εικόνα μικρότερη και να αλλάξω το 
φόρτωμα της εικόνας έτσι ώστε να γίνεται με opencv και όχι με pillow.
\end{document}
