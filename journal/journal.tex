\documentclass{article} 
\usepackage{polyglossia} 
\usepackage{amsmath}
\usepackage{fontspec} 
\usepackage{lipsum} 
\usepackage[margin=1in]{geometry}
\usepackage{graphicx} 
\usepackage{caption} 
\usepackage{subcaption}
\usepackage{hyperref} 
\usepackage{listing}
\hypersetup{% 
    colorlinks=true, linkcolor=blue, filecolor=magenta,      
    urlcolor=cyan, 
    pdfinfo = {%
        Title = Ημερολόγιο 2ης Εργασίας ΨΕΕ
        Author = {Χρήστος Μάριος Περδίκης},
        Producer = XeLaTeX,
    } 
}

\setmainfont{C059}
\pagestyle{headings}
\pagenumbering{roman}

\setlength{\parindent}{0pt}
\setlength{\parskip}{\medskipamount}


\title{Ημερολόγιο 2ης Εργασίας ΨΕΕ}
\date{Εαρινό Εξάμηνο 2024-2025}
\author{Χρήστος-Μάριος Περδίκης 10075 cperdikis@ece.auth.gr}

\begin{document}
\maketitle

\section{12/5/25 --- Η Αρχή Των Πάντων}
Διάβασα την εκφώνηση.

\section{13/5/25 --- Η Συνέχεια της Αρχής}
Ο προσωπικός βοηθός μου ο Κλαύδιος μου έδωσε το πρώτο draft κώδικα.
Μπορώ να πω οτι τρέχει απίστευτα αργά\ldots{} Πρόσθεσα μερικά μηνύματα για
να ξέρω πού βρίσκεται το πρόγραμμα κάθε φορά. Αύριο ας ακούσω τι έχει να πει ο
Κλαύδιος για τον κώδικα που μου έδωσε και ας τον δω να τρέχει μια φορά 
ολόκληρος. Έπειτα ας προσπαθήσω να υποδειγματοληπτήσω την εικόνα, ή έστω
να την κάνω χαμηλότερης ανάλυσης (αν έχουν διαφορά αυτά τα δύο, που
νομίζω δεν έχουν) και ας προσπαθήσω να αλλάξω το φόρτωμα της εικόνας
από pillow σε opencv.

\section{14/5/25 --- Σοβαρεύουν τα πράγματα}
Θα προσπαθήσω να κάνω την αρχική εικόνα μικρότερη και να αλλάξω το 
φόρτωμα της εικόνας έτσι ώστε να γίνεται με opencv και όχι με pillow.


Σμύκρινα την εικόνα στο μισό και ακόμα παίρνει πολύ ώρα\ldots{} Επίσης ο 
αλγόριθμος κύκλων Hough ανιχνεύει υπερβολικά πολλούς κύκλους, σίγουρα
είναι λάθος. Ας τον κοιτάξω λίγο μπας και αυξήσω κάποιο threshold.
Αλλιώς μπορεί να λάβω διαφορετικό thresh για το sobel edge πριν το 
circle detection. Μιας και το λέω τώρα, μήπως οι sobel\_edge, log\_edge
δεν θέτουν σωστά τη binary image που πρέπει να χρησιμοποιήσει η
circ\_hough;
Επίσης ο Laplacian of Gaussian τρόπος φαίνεται να έχει αρκετά κακά 
αποτελέσματα, ίσως είναι και αυτός λάθος.

Με ένα threshold στα zero crossings και ελέγχοντας όλα τα 8 πίξελ γύρω
από ένα κεντρικό, ο LoG edge detector είναι αρκετά καλύτερος. Μου
κάνει εντύπωση το gradient threshold όμως, γιατί παραγώγιση με central 
διαφορές; Γιατί όχι με forward διαφορές; Θα ήθελα να το διερευνήσω και αυτό.
Δουλεύει και με forward και με central διαφορές και με απλή διαφορά στα στοιχεία
του filtered\_image, βάζω πάλι central διαφορές γιατί μου φάνηκε το καλύτερο 
από τα υπόλοιπα.

Ας αρχίσω να βλέπω 
τι πάει λάθος με τη circ\_hough. Έχω τρεις επιλογές, να προσπαθήσω να διορθώσω
την τωρινή υλοποίηση, να ρωτήσω τον μαθητευόμενό μου τον Κλαύδιο να κάνει
νέα υλοποίηση ή να υλοποιήσω μια εγώ από την αρχή.

\end{document}
